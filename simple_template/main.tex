%% Presentation
\documentclass[aspectratio=169]{beamer}
\usepackage{pgfpages}
\mode<presentation>
% add notes
% \setbeameroption{show notes}
% \setbeameroption{show notes on second screen=right}
% \setbeamertemplate{note page}[compress]

%% handout
% \documentclass[aspectratio=169,handout]{beamer}
% \pgfpagesuselayout{6 on 1}[letterpaper,border shrink=5mm]
% \pgfpagesuselayout{4 on 1}[letterpaper,landscape,border shrink=5mm]



%Theme and Color of Presentation
\usetheme{CambridgeUS}
\usecolortheme{seahorse}
\useinnertheme{rectangles}

%Colors
% From https://www.utdallas.edu/brand/color-palette/
\definecolor{UTDorange}{RGB}{232,117,0}
\definecolor{UTDgreen}{RGB}{18,71,52}
\definecolor{UTDseafoam}{RGB}{95,244,183}

\setbeamercolor{palette primary}{bg=UTDorange,fg=white}
\setbeamercolor{palette secondary}{bg=UTDgreen,fg=white}
\setbeamercolor{palette tertiary}{bg=UTDorange,fg=white}
\setbeamercolor{palette quaternary}{bg=UTDorange,fg=white}
\setbeamercolor{frametitle}{fg=UTDgreen}
\setbeamercolor{structure}{fg=UTDgreen} % itemize, enumerate, etc
\setbeamercolor{subsection in head/foot}{bg=UTDgreen,fg=white}


%Additional Packages
\usepackage{graphicx}
\usepackage{physics}
\usepackage{amsmath}
\usepackage{setspace}
\usepackage{textpos}
\usepackage{hyperref}
\usepackage{xcolor}
% \usepackage{enumitem}

%Additional Settings/Commands
\newcommand{\extraspace}{\vskip 0.5em}

\AtBeginSection[]
{
	\begin{frame}<beamer>[noframenumbering]
		\frametitle{Outline}
		\tableofcontents[currentsection]
	\end{frame}
}
\AtBeginSubsection[]
{
	\begin{frame}<beamer>[noframenumbering]
		\frametitle{Outline}
		\tableofcontents[currentsection,currentsubsection]
	\end{frame}
}

%Presentation Info
\title[2nd-order System Dynamics]{Introduction to 2\textsuperscript{nd}-order System Response}
\author{Jonas Wagner}
\institute[UTDallas]{The University of Texas at Dallas}
\date{}

\begin{document}
	
\begin{frame}
	\titlepage
\end{frame}

\addtobeamertemplate{frametitle}{}{%
	\begin{textblock*}{100mm}(.85\textwidth,-1cm)
		\includegraphics[height=1cm]{logos/UT_Dallas_Logo}
	\end{textblock*}
}

\begin{frame}{Outline}
	\tableofcontents
	\note{
		4th-wall break notes
		\begin{itemize}
			\item Lecture Objective:
			% \begin{itemize}
				% \item 
				\textbf{why 2nd-order roots of a dynamical system's can result in more interesting transient dynamics}
				(i.e.) the 3 cases as a result from the quadratic equation
			% \end{itemize}
			\item Math background/assumptions:
			\begin{itemize}
				\item Simple ODEs solutions are covered in prereq and explained ealier in this course
				\item Laplace transform methods and the \textbf{inverse-laplace via partial fraction expansion} will be well known to students
				\item In a real course I'd spend time in lecture having students walk me through the various derivations instead of walking through them or leaving as an exercise/assignment.
			\end{itemize}
			\item Previous lectures:
			\begin{itemize}
				\item 1st order-system response and how time-constant plays into the system impulse and step-response
				\item Solutions to ODEs (w/in both time and frequency domains)
			\end{itemize}
		\end{itemize}
	}
\end{frame}

\section{Motivation}
\begin{frame}
	\frametitle{Real-World Dynamical Systems}
	\begin{columns}
		\begin{column}{0.6\linewidth}
			\centering
			\includegraphics[width=0.5\textwidth]{figs/LCR_circuit.png}\\
			\includegraphics[width=0.9\textwidth]{figs/car_supension.jpg}
		\end{column}
		\begin{column}{0.3\linewidth}
			\centering
			\includegraphics[width=0.9\textwidth]{figs/Quadcopter-model-and-the-coordinate-systems-relationship.png}
		\end{column}
	\end{columns}
\end{frame}

\section{Forced Response}
\begin{frame}
	\frametitle{Step Response - 1st vs 2nd order}
	\begin{itemize}[<+->]
		\item[] \textbf{Step Input:}
			$\textbf{u}(t) \overset{\mathcal{L}}{\Rightarrow} U(s) = \frac{1}{s}$\\
		\item[] \textbf{1\textsuperscript{st}-order:}
		\[
			Y(s) = \frac{K}{\alert<.(3)>{\tau s + 1}} \qty(\frac{1}{s})
			\quad \overset{\mathcal{L}^{-1}}{\Rightarrow} \quad
			y(t) = K (1 \alert<.(3)>{- e^{-t/\tau}}) \mathbf{u}(t)
		\]
		\only<+| handout:0>{\begin{center}\includegraphics[width = 0.3\textwidth]{figs/step_response_1st.jpg}\end{center}}
		\item[] \textbf{2\textsuperscript{nd}-order:}
		\[
			Y(s) = \frac{K}{\alert<+->{(s+p_1)(s+p_2)}} \qty(\frac{1}{s})
			% = \frac{C_1}{s} + \frac{C_2}{s+p_1} + \frac{C_3}{s+p_2}
			\quad \overset{\mathcal{L}^{-1}}{\Rightarrow} \quad
			y(t) = \qty(C_1 + \alert<.->{C_2 e^{-p_1 t} + C_3 e^{-p_2 t}}) \mathbf{u}(t)
		\]
	\end{itemize}
	\only<.| handout:0>{\begin{center}
		\alert{$\Delta(s)$ dictates transient dynamics}
	\end{center}}
	\only<+->{\begin{columns}
		\begin{column}{0.4\textwidth}
			\textbf{3 distinct cases:}
			\begin{itemize}
				\item<.-> Damped: \alert{$p_1 \neq p_2$}
				\item<+-> Critically Damped: \alert{$p_1 = p_2$}
				\only<.>{Special Case:$ 
					% C_2 e^{-p_1 t} + C_3 e^{-p_2 t} \to 
					\alert{C_2 e^{-p_{1,2} t} + C_3 t e^{-p_{1,2} t}}$}
				\item<+-> Underdamped: \alert{$p_{1,2} = \sigma \pm j \omega$}
			\end{itemize}
		\end{column}
		\begin{column}{0.4\textwidth}
			\centering
			\only<.(-2)| handout:0>{\includegraphics[width=0.5\columnwidth]{figs/step_response_2nd_overdamped.png}}
			\only<.(-1)| handout:0>{\includegraphics[width=0.5\columnwidth]{figs/step_response_2nd_critdamped.png}}
			\only<.| handout:0>{\includegraphics[width=0.5\columnwidth]{figs/step_response_2nd_underdamped.png}}
			\only<+->{\includegraphics[width=\columnwidth]{figs/step_response_2nd.png}}
		\end{column}
	\end{columns}}
\end{frame}

\section{Applied Example: Spring-Mass-Damper}
\subsection{Review: System Modeling}
\begin{frame}
	\frametitle{Spring Mass-Damper System Modeling}
	\begin{columns}
		\begin{column}{0.5 \textwidth}
			Newton's 2\textsuperscript{nd} Law:
			\[
				F = m \vb{a} 
				= m \dv{t} \vb{v} 
				= m \dv{t} \qty(\dv{t} \vb{x})
			\]
			\pause
			\[
				m \dv{t^2} x(t) 
				= \sum F 
				= f(t) - b \dv{t} x(t) - k x(t)
			\] 
			\vspace{1em}

			\pause{}
			Differential Equation: ($\vb{x} = x(t)$, $\vb{u} = f(t)$)
			% Let $\vb{x} = x(t)$ and $\vb{u} = f(t)$
			\[
				m \ddot{\vb{x}} + c \dot{\vb{x}} + k \vb{x} = \vb{u}
			\]
		\end{column}
		\begin{column}{0.375 \textwidth}
			\onslide<1->
			\begin{figure}
				\includegraphics[width=\textwidth]{figs/SpringMassDamper_cartSystem.png}
				Spring Mass Damper System \cite{ctms_engin_umich_SystemModeling}
			\end{figure}
		\end{column}
	\end{columns}
	
	\footnotesize{
		Activity: 
		\url{https://www.sccs.swarthmore.edu/users/12/abiele1/Linear/examples/simple.html}
	}
\end{frame}
\subsection{Derivation: Transfer Function and Step-Response}
\begin{frame}
	\frametitle{Transfer Function Derivation}
	\textbf{Convert Differential Equation to Laplace:}
	\alert<+>{($x(t) = \dot{x}(t) = 0$)}
	\[
		f(t) = m \ddot{x}(t) + b \dot{x}(t) + k x(t)
		\alert<.>{
			\quad \overset{\mathcal{L}}{\Rightarrow} \quad
		F(s) = m s^2 X(s) + b s X(s) + k X(s)}
	\]
	\only<+->{
		\textbf{Solve for $X(s)$ in terms of $F(s)$}
	}
	\only<.|handout:0>{
	\[
		F(s) = (ms^2 + b s + k) X(s)
	\]}
	\only<+->{
	\[
		X(s) = 
		\cfrac{1}{ms^2 + bs + k} F(s)
		\only<+>{= \cfrac{1}{\alert<.>{m}(s^2+\frac{b}{m}s+\frac{k}{m})}\alert<.>{\qty(\frac{k}{k})} F(s)}
		\only<+->{= \qty(\frac{1}{k})\cfrac{\frac{k}{m}}{s^2 + \frac{b}{m}s + \frac{k}{m}}  F(s)}
	\]}
	\only<+->{
	\textbf{Transfer Function:}
	\[
		H(s) = \frac{X(s)}{F(s)} = \alert<+>{\qty(\frac{1}{k})}
		\alert<+>{\cfrac{\frac{k}{m}}{s^2 + \frac{b}{m} s + \frac{k}{m}}}
		\only<.(-1)>{\Leftarrow\underset{\textbf{(Hook's Law)}}{\alert{F = k \Delta x}}}
		\only<.|handout:0>{\Leftarrow 
			\underset{\textbf{(Standard Form)}}{
				\alert{\frac{\omega_0^2}{s^2 + 2\zeta\omega_0 s + \omega_0^2}}
			}
		}
	\]
	}
\end{frame}

\begin{frame}
	\frametitle{Factoring the characteristic polynomial}
	Apply the quadratic formula to find the roots of the characteristic polynomial:
	\[
		\Delta(s) 
		% = s^2 + \frac{b}{m} s + \frac{k}{m}
		= m s^2 + b s + k
		\quad \Rightarrow \quad
		s = \cfrac{-b \pm \sqrt{b^2 - 4mk}}{2m}
	\]
	\pause
	\textbf{3 Potential cases:} %(may have learned in differential equations)
	\begin{enumerate}[<+- | alert@+>]
		\item \textbf{Damped}: 
		$b^2 > 4mk \Rightarrow 
		p_1 \neq p_2 \Rightarrow (s+p_1)(s+p_2)$
		% p_1 = a, p_2 = b \Rightarrow (s+a)(s+b)$
		% $(\frac{b}{2m})^2 > \frac{k}{m} \Rightarrow (s+a)(s+b)$
		\item \textbf{Critically Damped}:
		\(
			b^2 = 4mk \Rightarrow p_1 = p_2 
			\Rightarrow {(s+p_{1,2})}^2
		\)
		% p_1 = p_2 = a \Rightarrow (s + a)^2$
		% $(\frac{b}{2m})^2 = \frac{k}{m} \Rightarrow (s + a)^2$
		\item \textbf{Underdamped}: 
		$b^2 < 4mk \Rightarrow p_{1,2} = \sigma \pm j \omega 
		\Rightarrow (s+\sigma\pm j\omega) = ((s+\sigma)^2 + \omega^2)
		$
		% \Rightarrow (s+\sigma +j\omega)(s + \sigma - j\omega)$
		% $(\frac{b}{2m})^2 < \frac{k}{m} \Rightarrow (s + \sigma \pm j \omega)$
	\end{enumerate}
	
	\note<+>{
	This motivates the standard characteristic polynomial form:
	\begin{align*}
		s^2 + 2 \zeta \omega_0 s + \omega_0^2
		\Rightarrow
		s = \zeta \omega_0 \pm \sqrt{(\zeta \omega_0)^2 - \omega_0^2}
		= \omega_0 \qty(\zeta \pm \sqrt{\zeta - 1})\\
		\intertext{Let $2 \zeta \omega_n = \sqrt{\frac{b}{m}}$ and $\omega_0 = \sqrt{\frac{k}{m}}$}
		\Delta(s) = s^2 + \frac{b}{m} s + \qty(\sqrt{\frac{k}{m}})^2
		\iff \Delta(s) = s^2 + 2\zeta \omega_0 s + \omega_0^2
	\end{align*}
	In this instance, the three cases are easily seen based on $\zeta$:
	\begin{enumerate}
		\item Damped: $\zeta > 1$
		\item Critically Damped: $\zeta = 1$
		\item Underdamped: $\zeta \in [0,1)$
	\end{enumerate}
	}
\end{frame}

\subsection{Derivation/Activity: Response Comparison}
\begin{frame}
	\frametitle{Case 1 (Damped)}

	\textbf{Distinct real roots:}
	$p_1 \neq p_2 \Rightarrow \Delta(s) = (s+p_1) (s+p_2)$

	\pause{}
	\[
		X(s) = \qty(\frac{1}{k}) \cfrac{\frac{k}{m}}{s(s^2 + \frac{b}{m} s + \frac{k}{m})}
		= \cfrac{K}{s(s+p_1)(s+p_2)}
	\]
	% \[
	% 	X(s) = \qty(\frac{1}{k}) \cfrac{\frac{k}{m}}{s^2 + \frac{b}{m} s + \frac{k}{m}}
	% 	= \qty(\frac{1}{k}) \cfrac{\frac{k}{m}}{s(s+a)(s+b)}
	% \]
	\pause{}

	\textbf{Partial Fraction Expansion:} %\textbf{(derive as homework)}
	\[
		X(s) = \cfrac{C_1}{s} + \cfrac{C_2}{s+p_1} + \cfrac{C_3}{s+p_2}
	\]
	\pause{}
	\textbf{Inverse Laplace:}
	\[
		\overset{\mathcal{L}^{-1}}{\Rightarrow}
		x(t) = \qty(C_1 + C_2 e^{-p_1t} + C_3 e^{-p_2t}) u(t)	
	\]

	\note<.>{
		% \[
		% 	C_{1,2,3} = \eval{\cfrac{\frac{1}{m}}{s(s+a)(s+b)}(s-\lambda_{i})}_{s = \lambda_{i}}
		% \]
		Let $a = \frac{b}{2m} + \sqrt{\qty(\frac{b}{2m})^2 - \qty(\sqrt{\frac{k}{m}})^2}$ and $b = \frac{b}{2m} - \sqrt{\qty(\frac{b}{2m})^2 - \qty(\sqrt{\frac{k}{m}})^2}$
		Evaluate coeficients:
		$(a)(b) = (\frac{b}{2m})^2 - ((\frac{b}{m})^2-\frac{k}{m}) = \frac{k}{m}$, \quad $(a-b) = 2\sqrt{(\qty(\frac{b}{2m})^2 - \frac{k}{m})}$
		\begin{gather*}
			C_1 = \eval{\frac{(s)}{m s(s+a)(s+b)}}_{s=0}
			= \frac{1}{m (a)(b)}
			\Rightarrow
			C_1 = \frac{1}{k} \textbf{(Hook's Law @ steady-state)}
			\\
			C_2 = \eval{\frac{(s+a)}{m s(s+a)(s+b)}}_{s=-a}
			= \frac{1}{m(-a)(-a+b)}
			= \frac{1}{m(a)(a-b)}
			\\
			C_3 = \eval{\frac{(s+b)}{m s(s+a)(s+b)}}_{s=-b}
			= \frac{1}{m(-b)(a-b)}
			= \frac{-1}{m(b)(a-b)}\\
			C_{2,3} = \cfrac{\pm 1}{2m\sqrt{(\qty(\frac{b}{2m})^2 - \frac{k}{m})} \qty(\frac{b}{2m} \pm \sqrt{\qty(\frac{b}{2m})^2 - \qty(\sqrt{\frac{k}{m}})^2})}
		\end{gather*}
	}
	
\end{frame}

\begin{frame}
	\frametitle{Case 2 (Critically Damped)}


	\textbf{Repeated Roots:}
	\(
		b^2 = 4mk \Rightarrow p_1 = p_2 
		\Rightarrow \Delta(s) = {(s+p)}^2
	\)

	\pause{}
	\[
		X(s) = \qty(\frac{1}{k}) \cfrac{\frac{k}{m}}{s(s^2 + \frac{b}{m} s + \frac{k}{m})}
		= \cfrac{K}{s (s+p)^2} 
	\]
	% \[
	% 	X(s) = \qty(\frac{1}{k}) \cfrac{\frac{k}{m}}{s^2 + \frac{b}{m} s + \frac{k}{m}}
	% 	= \qty(\frac{1}{k}) \cfrac{\frac{k}{m}}{s(s+a)(s+b)}
	% \]
	\pause{}

	\textbf{Partial Fraction Expansion:} %\textbf{(derive as homework)}
	\[
		X(s) = \cfrac{C_1}{s} + \cfrac{C_2}{s+p} + \cfrac{C_3}{(s+p)^2}
	\]
	\pause{}
	\textbf{Inverse Laplace:}
	\[
		\overset{\mathcal{L}^{-1}}{\Rightarrow}
		x(t) = \qty(C_1 + C_2 e^{-pt} + C_3 t e^{-pt}) u(t)	
	\]




	% % Let $a = \frac{b}{2m}$
	% \textbf{Repeated Roots:}
	% \(
	% 	b^2 = 4mk \Rightarrow p_1 = p_2 
	% 	\Rightarrow \Delta(s) = s{(s+p_{1,2})}^2
	% \)
	% \begin{gather*}
	% 	X(s) = \cfrac{K}{s (s+p_{1,2})^2} 
	% 	= \cfrac{C_1}{s} + \cfrac{C_2}{s+p_{1,2}} + \cfrac{C_3}{(s+p_{1,2})^2}\\
	% 	\overset{\mathcal{L}}{\Rightarrow}
	% 	x(t) = \qty(C_1 + C_2 e^{-p_{1,2} t} + C_3 t e^{-p_{1,2} t}) u(t)
	% \end{gather*}
\end{frame}

\begin{frame}
	\frametitle{Case 3 (Underdamped)}

	\textbf{Complex Roots:}
	\(
		b^2 < 4mk \Rightarrow p_{1,2} = \sigma \pm j \omega 
		\Rightarrow \Delta(s) = {(s+\sigma\pm j\omega)}^2 \only<+|alert@+>{= ((s+\sigma)^2 + \omega^2)}
	\)

	\pause{}
	\[
		X(s) = \qty(\frac{1}{k}) \cfrac{\frac{k}{m}}{s(s^2 + \frac{b}{m} s + \frac{k}{m})}
		= \cfrac{K}{s (s+\sigma \pm j\omega)}
	\]
	% \[
	% 	X(s) = \qty(\frac{1}{k}) \cfrac{\frac{k}{m}}{s^2 + \frac{b}{m} s + \frac{k}{m}}
	% 	= \qty(\frac{1}{k}) \cfrac{\frac{k}{m}}{s(s+a)(s+b)}
	% \]
	\pause{}

	\textbf{Partial Fraction Expansion:} %\textbf{(derive as homework)}
	\[
		X(s) = \cfrac{C_1}{s} + \cfrac{C_2}{(s+\sigma+j\omega)} + \cfrac{C_3}{(s+\sigma - j\omega)}
	\]
	\pause{}
	\textbf{Inverse Laplace:}
	\begin{align*}
		\overset{\mathcal{L}^{-1}}{\Rightarrow}
		x(t) &= \qty(C_1 + C_2 e^{-\sigma t}e^{j\omega t} + C_3 e^{-\sigma t} e^{-j\omega t}) u(t)\\
		\only<+->{&=C_1 u(t) + 2 e^{-\sigma t} \alert<+>{\qty(\cfrac{C_2 e^{j\omega t} + C_3 e^{-j\omega t}}{2})}  u(t) 
		\visible<.>{\alert{\Leftarrow \textbf{Convert using Euler's Identity}}}}
	\end{align*}



	% % Let $\sigma = \frac{b}{m}$ and $\omega = \sqrt{\sqrt{\frac{k}{m}}^2 - \qty(\frac{b}{2m})^2}$
	
	% \[
	% 	X(s) = \cfrac{K}{s (s+\sigma \pm j\omega)}
	% \]

	% \begin{align*}
	% 	X(s) &= \cfrac{\frac{1}{m}}{s (s+\sigma \pm j\omega)} 
	% 	= \cfrac{C_1}{s} + \cfrac{C_2}{(s+\sigma+j\omega)} + \cfrac{C_3}{(s+\sigma - j\omega)}\\
	% 	& \qquad \Updownarrow \mathcal{L}\\
	% 	% \overset{\mathcal{L}}{\Rightarrow}\\
	% 	% \overset{\mathcal{L}}{\Rightarrow}
	% 	x(t) &= \qty(C_1 + C_2 e^{-\sigma t}e^{j\omega t} + C_3 e^{-\sigma t} e^{-j\omega t}) u(t)\\
	% 	&= C_1 u(t) + 2 e^{-\sigma t} u(t) \alert<+>{\frac{C_2 e^{j\omega t} + C_3 e^{-j\omega t}}{2}}
	% 	\only<.->{\alert{\Leftarrow \textbf{Convert using Euler's Identity}}}
	% \end{align*}
	

\note<.>{
	Alternative approach
	\begin{align*}
		X(s) &= \frac{K}{s(s+\sigma\pm j\omega)}
		= \frac{K}{s((s+\sigma)^2 + \omega^2)}\\
		&= \frac{C_1}{s} + \frac{C_2}{(s+\sigma)^2 + \omega^2}
		\overset{\mathcal{L}^{-1}}{\Rightarrow}\\
		&\overset{\mathcal{L}^{-1}}{\Rightarrow}
		x(t) = \qty(C_1 + \frac{C_2}{\sigma}e^{-\sigma t} \cos(\omega t))
	\end{align*}

	Specific case:
	\begin{align*}
		X(s) &= \cfrac{\frac{1}{m}}{s (s^2 + \frac{b}{m}s + \frac{k}{m})}
		= \cfrac{C_1}{s} + \cfrac{C_2}{(s+\frac{b}{2m})^2 + \sqrt{\frac{k}{m}-\sqrt{\frac{b}{m}}}^2}
		\overset{\mathcal{L}^{-1}}{\Rightarrow}\\
		&\overset{\mathcal{L}^{-1}}{\Rightarrow}
		x(t) = \qty(C_1 + \frac{C_2}{\sqrt{\frac{k}{m}-\sqrt{\frac{b}{m}}}} \exp{-\cfrac{b}{2m}t} \cos{\qty(\qty(\sqrt{\frac{k}{m}-\sqrt{\frac{b}{m}}})t)}) u(t)
	\end{align*}
}

\end{frame}

\begin{frame}
	\frametitle{Activity: Response Comparison}

	TODO:
	\begin{enumerate}
		\item Experiment with different $m$,$k$, and $b$ parameters to gain an intuitive understanding of how each parameter effects the response
		\item Select one of each case and derive the functional form (i.e. solve for poles and coefficients)
	\end{enumerate}

	\extraspace
	Online Tool:
	\footnotesize{\url{https://www.sccs.swarthmore.edu/users/12/abiele1/Linear/examples/simple.html}}

\end{frame}

\section*{}

\begin{frame}
	\frametitle{Lecture Overview}
	\[
		m \ddot{x}(t) + b \dot{x}(t) + k x(t) = f(t)
	\]
	\[
		X(s) = \frac{1}{m s^2 + b s + k} F(s) 
		= \frac{\frac{k}{m}}{s^2 + \frac{b}{m} s + \frac{k}{m}} \frac{1}{k} F(s)
	\]
	\begin{columns}
		\begin{column}{0.5\textwidth}
			\[
				H(s) = \frac{X(s)}{F(s)}%{U(s)}
				= \qty(K)\frac{
					\omega_0^2
				}{
					s^2 + 2 \zeta \omega_0 s + \omega_0^2
				}
			\]		
	\[
		\omega_0 = \sqrt{\frac{k}{m}}
		\quad
		\zeta = \sqrt{\frac{b^2}{4 m k}}
		\quad
		K = \frac{1}{k}
	\]
			% \[
			% 	\omega_0 = \sqrt{\frac{k}{m}}
			% 	\quad
			% 	\zeta = \sqrt{\frac{c^2}{4 m k}}
			% \]
		\end{column}
		\begin{column}{0.25 \textwidth}
			\begin{figure}[]
				\includegraphics[width=\textwidth]{figs/SpringMassDamper_cartSystem.png}
			\end{figure}
		\end{column}
		% \begin{column}{0.3\textwidth}
		% 	\[
		% 		\color{red}
		% 		u(t) = U_0 \cos(\omega t)
		% 	\]
		% 	\[
		% 		\color{blue}
		% 		y(t) = Y_0 \cos(\omega t + \phi)
		% 	\]
		% 	\begin{figure}
		% 		\includegraphics[width=\textwidth]{figs/phase_shift.png}
		% 	\end{figure}
		% \end{column}
		% \begin{column}{0.4\textwidth}
		% 	\begin{figure}
		% 		\includegraphics[width=\textwidth]{figs/bode_demo_2.png}
		% 	\end{figure}
		% \end{column}
	\end{columns}
\end{frame}


\begin{frame}[allowframebreaks,noframenumbering]{Bibliography}
	\bibliographystyle{unsrt}
	\bibliography{refs}
\end{frame}



\begin{frame}[noframenumbering]
	\frametitle{TLDR: Second-Order System Dynamics}
	\begin{columns}
		\begin{column}{0.5\textwidth}
			\textbf{Transfer Function}
			\[
				H(s) 
				= \cfrac{Y(s)}{U(s)} 
				= \cfrac{
					\omega_0^2
				}{
					s^2 + 2 \zeta \omega_0 s + \omega_0^2
				}
			\]
			\textbf{System Poles}
			\[
				s = - \zeta \omega_0 \pm \omega_0 \sqrt{1 - \zeta^2}
			\]
			\textbf{Spring Mass Damper System Parameters}
			\[
				\omega_0 = \sqrt{\cfrac{k}{m}}
				\hspace{0.5 in}
				\zeta = \sqrt{\cfrac{c^2}{4 m k}}
			\]
		\end{column}
		\begin{column}{0.375\textwidth}
			\begin{figure}[]
				\includegraphics[width=\textwidth]{figs/2ndOrderTransient.png}
				2nd Order System Response \cite{engineerOnADisk_2ndOrderDynamics}
			\end{figure}
		\end{column}
	\end{columns}
\end{frame}

\end{document}